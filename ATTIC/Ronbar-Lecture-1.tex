
Rachel walked to the blackboard. ``I would like to start,'' she said, ``with a simple
question.'' She drew a line and a point, labeling each. ``In the plane of this blackboard, given
a straight line L and a point P that is not on L, how many straight lines can be drawn through P
parallel to L?''

She put down the chalk and turned to face the audience, waiting. Nobody said a word. Colty
shifted in his seat and clicked his teeth slightly. He wondered what the great scientists
assembled there were thinking. Rachel's question was entirely elementary, almost condescending
in its simplicity, yet Colty knew her well enough to know that she undoubtedly had a deeper
purpose behind it.

``What is your answer?'' she asked. Her voice was soft, yet direct. She expected an answer.

``One, obviously,'' someone said on the other side of the room from Colty. He wasn't sure
who it was.

``One,'' she repeated. She went to the board and in the corner wrote ``There is one straight
line through P parallel to L.'' She turned again and waited. ``Does anyone have a different
answer?'' There was some impatient shifting.

``Now let me ask my real question\ldots'' She paused. ``\emph{How do you know?}''

Again there was silence.

``It needs no explanation,'' said Tomtus. ``It is a self-evident basis from which we can
begin our reasoning.''

``Yet that doesn't make it true,'' she said. ``It only means we hope it is true.''

``It can be experimentally supported,'' said another voice.

She drew another line on the board this time through P that was approximately parallel to L.
``Here is the one parallel line you speculate about. Now let me draw another.'' She then drew a
second line through P but at an angle to the first one. She drew it short so that it did not
actually touch the original line L. ``This new line,'' she said, ``does not appear to intersect
with L. Perhaps it is also parallel to L .''

``Of course it isn't,'' said Tomtus. Colty glanced over the audience. Chortwal looked
displeased.

Rachel pretended to ignore his last comment. ``How can you tell that it's not parallel to
L?'' she asked.

``This is silly,'' said Tomtus. ``I'm sure you know how. Just extend it.''

Rachel went to the board again and drew extra lengths on the angled line. Of course it did
intersect L as predicted. ``You are right.''

The audience murmured as she drew another line, this time angled less than before. ``Will this
new line also intersect L?'' she asked.

Silence.

This time when she extended the line she encountered the edge of the blackboard before it
reached L. Yet it was obvious from the angle that it would intersect L at a point maybe about
half way to the doorway of the hall.

``It looks like it probably will intersect.'' She stepped away from the board and, facing the
audience, held her arm out in a way only a chortak could and pointed toward the figure she had
drawn. ``But what if I drew a line so close to the parallel one that its point of intersection
with L, as you might predict, was in the next room\ldots\ or the next province\ldots\ or the
next galaxy? Would it still intersect in reality?''

``Of course,'' said Tomtus. ``It is the power of mathematics that we can predict these things
that we can't directly measure.''

She turned toward him and cocked her head slightly. ``\emph{How do you know?}''

Silence.

``No simple experiment can answer this question in the general case,'' she said.

``You don't need an experiment,'' said Hortnes. ``It can be proved.''

``Ah yes\ldots\ proof,'' said Rachel. She held out the chalk between her flexible fingers and
said, ``I invite you to come forward and show me this proof.''

Hortnes pretended to get up thinking that would be enough, but to his surprise Rachel insisted.
``No, really,'' she said, ``please show me.'' With some reluctance Hortnes went down to the
floor. Rachel held out the chalk and he hesitated. Sensing his reticence Rachel put the chalk on
the tray and Hortnes picked it up and started writing on the board. Rachel watched as Hortnes
then worked a standard proof.

``Please explain it to me as you do it,'' Rachel said.

Hortnes seemed amused by this but humored her. ``It's really very simple, he commented. This is
a proof that we have our children do in their first geometry class.'' Rachel seemed completely
unperturbed by the implication that she wasn't as smart as dunari children but the association
was not lost by the rest of the audience. Chortwal now looked both bored and angry.

When Hortnes was finished he put the chalk down and announced, ``Thus it is so.'' Then he moved
to sit down.

``Wait,'' Rachel said.

She silently studied the proof for several minutes. Nobody made a sound. Finally, she pointed
one of her long fingers at a step in the middle of the proof. ``What is the justification for
this step?'' she asked.

Hortnes started to explain but she interrupted him. ``Doesn't that require you to assume
the conclusion you are trying to prove?''

``How so?'' he asked.

Rachel took the chalk and started to work on the board. She expanded the step under discussion
into a number of more primitive sub-steps, talking about the justification of each. ``Here,
though,'' she said, ``we need to assume that this parallel line is unique. That's the
very thing we are trying to prove.''

A shiver passed over Colty's body.

Hortnes stared at her. Then he looked at the board. ``Well\ldots''

``Yes?'' Rachel asked.

``Um. I see what you mean,'' he finally admitted.

``So the proof is invalid?'' Rachel asked.

``Interesting.''

Hortnes returned to his seat as Rachel explained in detail about the logical flaws in a proof
that the dunari had accepted for centuries. Colty's heart pounded. He wondered where this was
going.

When she was finished Chortwal raised his trunk. ``Dr. Spencer,'' he said simply. ``I have a
correct proof. If I show it to you can we move past this silliness?''

Rachel seemed almost excited. ``Yes, please come and show us.''

Chortwal went down to the floor and erased Hortnes's ``proof.'' He then presented a much more
elaborate and detailed proof in its place. Colty saw that it was based on Parmolic Normalization
but Chortwal used a variation of the technique that Colty never saw before. Considering
Chortwal's reputation for innovative proof techniques, it wouldn't have surprised Colty if
Chortwal had just thought of it.

Rachel watched the proof carefully but didn't make any comments while Chortwal worked. For
almost a half hour, Chortwal held the floor and displayed his mathematical prowess during
Rachel's lecture time. If Rachel had been dunari it would have been a humiliating experience for
her. Yet she didn't seem to notice.

``And thus it is so!'' Chortwal finally finished. He turned to Rachel. ``I will return to my
seat now.''

``Wait,'' Rachel said.

She studied the proof for quite some time and nobody said a word. For a while Rachel closed her
eyes. Chortwal looked pleased. ``It is quite correct,'' he assured her, ``there is no point
searching for an error that's not there.''

Rachel walked to the board and wrote some symbols in an alien script. She studied them and then
erased them. She drew another line on his figure and then wrote some more symbols in her strange
language. They weren't satisfactory either.

``Your proof is quite ingenious,'' she said finally. ``I admit that I can't find a flaw. I will
need assistance. In the meantime, please have a seat. Thank you for your proof.'' As Chortwal
made his way back up the aisle, Rachel took out her machine and unfolded it. They would see some
dazzling human technology after all.

``I don't normally like to use this device here, but Dr. Chortwal's proof is too good for me. I
will let my machine consider this problem as well.'' She held up the device, facing the board,
the way a person might take a picture with a camera. Then she touched the surface of her machine
for a while, working with it. From Colty's vantage point he couldn't really see what she was
doing.

After a few minutes she put the device on the podium and then faced the audience. In a clear
voice she explained, ``My machine is now considering every possibility relevant to this problem,
searching for a case where the logic of Dr. Chortwal's proof fails. It is searching far faster
than I can think using a technique called `model checking' that is standard in our engineering
disciplines for verifying the correctness of complex systems.''

Rachel paused a moment and glanced at her machine. ``This might take a while so in the meantime
I'd like to continue my lecture.''

At that moment, however, there was a soft beep and then a voice spoke in the human language. It
was a disconcerting sound that seemed to come from nowhere, but Rachel was not startled. ``Ah,''
she said. ``A counter example has been found already.''

She picked up her machine and spoke back to it in the strange, chittering language that humans
use and the two of them had a short conversation. The mood in the room was subdued. This was
technology far beyond anything dunari. As silly and simple minded as Rachel sometimes seemed,
her machine, and the way she used it, was a stark reminder of just where the dunari stood
relative to their human visitors.

Rachel studied the machine for a time and then smiled. ``Wonderful,'' she said softly in
Argenian. Then in a louder voice she addressed Chortwal who was sitting three quarters of the
way toward the back. ``Very ingenious, Doctor,'' she said, ``but not quite correct.
Observe\ldots''

Rachel went to the board and dived in deeply into an intricate and detailed explanation of the
flaw. It wasn't easy to see. It wasn't easy to understand. Yet with her machine in one
hand, which she consulted often at first, Rachel dissected the reasoning into small,
unassailable steps. Several people in the audience got involved, sometimes challenging her,
sometimes assisting her.

Colty surprised himself by raising his trunk. ``Can't you work around the basic problem by
invoking the Barapitus theorem?''

``Barapitus, I understand,'' Rachel replied. ``That won't work because the Barapitus
theorem requires there to be an equivalence relation between the three witness points. That
isn't the case here.''

``Yes\ldots\ I see,'' Colty said. ``You're right. Thank you.''

Rachel seemed particularly interested in making sure everyone understood her explanation. Her
lecturing style was quite interactive and she often asked questions to members of the audience
to make sure they were paying attention. Finally she asked Chortwal, who had remained silent
during the entire proceeding, ``Do you agree that the proof is invalid?''

Chortwal didn't answer at first. Then he shifted a bit in his seat uncomfortably. ``I'd have to
take a closer look at your argument,'' he said finally. ``I could probably fix any flaws that
remain.''

``I will give you my notes,'' Rachel said, ``so you can study them at your leisure.''

Yet even after all that she wouldn't let the basic topic go.

``The claim has been made that there is only one though P parallel to L,'' she said. ``Yet that
can't be verified by experiment nor do we have a proof of it. I ask again\ldots \emph{How do you
  know?}''

The discussion continued throughout the afternoon exploring this one simple question. Other
suggestions for proofs were made, but Rachel quickly dismantled them. She didn't even need her
machine. The issue of experimental verification was raised again and more sophisticated
experiments were suggested. Yet those too failed. Colty sat in amazement as the greatest minds
in Argenia struggled to come up with a convincing argument for such a simple question. After
each attempt Rachel asked again, ``How do you know?'' As the day wore on the impatience in the
room increased.

Finally Chortwal couldn't take it any longer. ``I don't understand why we are spending so much
time on this absurd question,'' he started. ``Maybe it is different for you humans, but among
dunari this matter is obvious. A child can understand it.'' Chortwal raised his voice and was
obviously very agitated. ``I set aside my research to come here hoping to gain some profound
insight into the nature of the universe, and instead I am being lectured to about elementary
geometry. What are you trying to say, Dr. Spencer?'' He waved his trunk toward the statement
still written on the board. ``Are you saying that statement is false?''

The room fell silent.

Rachel took a step in Chortwal's direction. ``I hear frustration in your voice,'' she said
calmly. ``Indeed we have been discussing this question in detail and at great length. Perhaps it
is time for me to speak plainly.'' She raised her arm and pointed at the statement scrawled on
the board. In a loud, commanding voice unlike any Colty had ever heard her use before she said,
``\emph{That statement is false}.'' She paused and then continued in her normal voice. ``The
true answer is zero. \emph{Parallel lines do not exist}.''

Silence.

She lowered her arm and crossed her hands in front of her. Then in a soft voice she continued.
``And now you have a question for me,'' she said. ``What is that question?''

A moment passed. And then another. Nobody said a word. Then, suddenly from the back of the room
where the students were seated came a loud voice. It was Denart.

``\emph{How do you know?}''

``Excellent!'' Rachel cried out. She faced the students, eyes wide with excitement, and gave
them a nod of approval. Then she turned and faced the rest of the audience and, in her normal
voice, repeated their question. ``How\ldots\ do I know?''

``A complete answer to that question is not easy to give,'' she continued. ``\emph{But a
complete answer you shall have}. By the time we are done with this lecture series you too will
know this thing as I do. In the coming lectures I will develop the mathematical machinery needed
to address this question properly. In the process we will explore the limits of logic and
reason, and you will see many fascinating and surprising results.''

Colty felt his skin crawl with excitement.

% TODO: Rachel gets a little preachy here. Simplify the rest of this scene.

``Dr. Chortwal,'' Rachel said. Her voice was no longer soft and timid but instead full and
strong. ``You said you were hoping to gain profound insight into the nature of the universe.
Let me assure you that you won't be disappointed. I ask you to bear with me and to remain
open minded.'' Chortwal nodded slightly.

``You see,'' Rachel continued. ``Today's lesson isn't just some fact about geometry. Oh no.
Today's lesson is about something far more important. It's about doubt. You must learn to doubt.
Over the past several seasons I've studied your mathematics and your physics. You have many fine
results and many ideas that seem fresh and innovative to me. Yet you are also stuck. For
hundreds of years you have made no progress against certain topics\ldots\ topics that you have
come to suspect might be beyond dunari understanding.''

Rachel was very animated now. She paced across the floor and moved her hands to emphasize her
words.

``I believe, however, that are you not as limited as you fear. Yet to move beyond the
barriers, you must learn to question your assumptions. Things are not as they seem and some very
basic things that you believe to be true are not.''

``I know it isn't easy. It wasn't easy for us either.'' She pointed at the board again. ``For
thousands of years our mathematicians also believed that there was exactly one line through P
parallel to L. After all\ldots\ it is obvious. The truth was difficult to accept. I know this
truth because I am the product of generations of mathematical thought. It will take time for you
to also fully understand.''
