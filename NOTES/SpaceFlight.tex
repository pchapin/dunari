% Local IspellPersDict: ../aspell.dunari.pws

\chapter{Space Flight}

When the humans started to look for a location for their new telescope they wanted a spot at the
edge of the galaxy where they could get a good view of intergalactic space without the obscuring
effects of interstellar gas and dust. Unfortunately from Earth it is about 20,000 light years to
the rim of the galaxy. That is too great a distance to mount a major construction project.
However, the galactic disk is only about 1000 light years thick. It happens that Earth is very
close to the galactic equatorial plane (only about 50 light years away). Thus by going
perpendicular to the galactic disk we can reach the top or bottom surface of that disk in only
500 light years.

The humans first launched some automated probes to scout for suitable locations. These probes
are smaller and lighter than ships that can carry people, and can thus move faster. After
finding a few candidate locations, a small team used \textit{Summer Breeze} to make a follow up
study of the most promising spots. It was during one of those follow up studies that Rujaria was
discovered (the probe that originally surveyed the system did not notice Rujaran society).

Rujaria is located 587 light years from Earth. It takes \textit{Summer Breeze} 170 days (a
little less than six months) at full speed to cover that distance. Thus the top speed of
\textit{Summer Breeze} is about 1260x the speed of light. At that rate it only takes about 16
seconds to travel from Earth to Pluto (using Pluto's ``average'' distance from the sun).
\textit{Summer Breeze} is able to cover one light year in just under seven hours, requiring 30
hours to travel from Earth to Alpha Centauri (4.3 light years away).

When traveling away from Earth, \textit{Summer Breeze} will cover 500 AU in about 200 seconds.
At that distance the sun will had faded to the brightness of the full moon (as seen from Earth).
Thus an observer watching the sun as \textit{Summer Breeze} starts its journey to Rujaria would
be able to see the sun fade as it recedes into the distance. In 22 hours \textit{Summer Breeze}
will have moved far enough away from the sun to dim our home star to magnitude zero, about equal
to the brightest stars in the sky.

To a traveler on the spacecraft the stars would not whiz across the sky (as they do on
\textit{Star Trek}), but one could probably watch the nearby stars moving slowly. Certainly
after a few hours there would be noticeable changes in the shape of constellations.

\textit{Summer Breeze}, and all human interstellar spacecraft, use a ``spacial escape''
technology to enable faster than light travel. Essentially the spacecraft is temporarily
disconnected from the universe. From the spacecraft's point of view there is no acceleration nor
any motion; in effect the universe moves around the spacecraft instead of the spacecraft moving
through the universe. Thus when the engines are engaged the ``acceleration'' to full speed is
nearly instantaneous. Yet the passengers feel no sensation of movement (they do feel an odd
``floating'' sensation when the spacial disconnect happens).

One happy consequence of this technology is that a moving spacial escape vehicle can travel
through objects in the normal universe. Extremely dense or hot materials, or strong
gravitational fields, can cause problems. Yet a craft like \textit{Summer Breeze} is able to
pass through rocks, asteroids, and even small planetoids unaffected. The humans have built
specialized craft that can penetrate into the core of stars but a large ship like \textit{Summer
  Breeze} is not able to do that.

When a spacial escape vehicle arrives at a planetary system it requires fine control to bring the
vehicle to a stop at an appropriate time/place. When moving at 1260x the speed of light the time
needed for ``tight'' maneuvering, such as is required in the vicinity of plants, is very short.
Since the environment around Earth is busy, spacecraft are generally directed to stop some
distance away, and then approach Earth using more conventional (and slower) drive technology.
Commercial ships are towed but \textit{Summer Breeze}, being a research vessel, does have
conventional drives of its own since it goes places where there are no tugs. Although in the
wilderness, it is considered safe to use spacial escape drives even fairly close to major
planets.


