% Local IspellPersDict: ../aspell.dunari.pws

\chapter{Rujar}

Rujar orbits an M class red dwarf with an orbital period of about 37 days. It is tidally
locked so that one side always faces the Rujaran sun while the other side is in perpetual
night. This means it has a rotational period of 37 days as well. It has a mass of 0.83 Earths
and a surface gravity that is 0.9G (where Earth is 1.0G). Atmospheric oxygen is only 17\%,
making it difficult for humans to do heavy labor there.

Rujar's orbital period is not significant to the dunari because during most of their
civilization they were completely unware of it. Instead their calander is connected to the solar
cycle of the Rujaran sun. This is about the same as an Earth year (note that in real life
Proxima Centurai's solar cycle is about 440 Earth days). This time is called a ``runion.'' The
sun produces many flares (called ``flashes'' by the dunari) during the ``winter'' season.

Rujar has less water than Earth and much of its water is locked up in large glaciers on the
cold, night side of the planet. These glaciers melt as they move into the sun and rivers flow
from them onto the day side. Several seas exist on the day side, particularly near the boundary
between day and night, called the ``Narlarken'' by the dunari. The heat from the Rujaran sun
evaporates water off these seas and steady winds, called the ``Nermella'' and ``Nermellum''
(driven by the temperature difference between day and night), circulate that vapor back to the
night side where it rains/snows. Rain is rare on the day side except close to the Narlarken.

Most biological diversity on Rujar is close to the Narlarken where temperatures are moderate
and liquid water is plentiful. Light loving life forms are complemented by nocturnal forms that
live just inside the darkness. However, conditions far from the Narlarken are harsh. Much of the
day side is a desert with temperatures well over 150 degrees F in many places. On the deep night
side, temperatures can drop to well below -100 degrees F. Very few life forms can endure either
extreme but many Rujaran species are highly adapted to hot, dry conditions in order to make
living away from the Narlarken feasible.

Dunari society mostly lives on the day side in a band about 1500 miles wide along the Narlarken
and concentrated around the lush, tropical regions near the equator. They have explored most of
the day side but exploration of the deep night has only recently started with the invention of
long distance aerocars (airplanes). Dunari astronomy is relatively primitive because they've had
little opportunity to see the stars living as they do in perpetual day (and with Rujar's often
dusty atmosphere). However, the existence of the stars has been know since ancient times. Many
explorers have ventured into the ``near night'' at various times in dunari history. At the time
of the human arrival there were some observatories in the ``middle night'' region of the planet
(the dunari version of a space telescope... expensive, high tech, etc).

The dunari are evolved from an elephant-like creature. They are quadrupeds that use a flexible
trunk for manipulating their environment much as humans use their hands. However, they are not
mammals in the Earth sense of the word. They are highly adapted to desert life, can tolerate
high temperatures, and need very little water. They are covered with scales and are ``cold
blooded'' similar to Earth reptiles. As a consequence they also consume much less food per unit
of body weight than humans. They also aren't fond of cold or dark.

Physically the dunari are larger than humans. They are not as tall (about 4 feet, with 5 feet
being unusual) but they are long and bulky. A typical dunari weighs 50\% more than a typical
human. They don't sit as humans do but instead lay down (think: horses). They have eyes on
either side of their head that do not have overlapping fields of vision. They have the instincts
of a herd animal; for example, they tend to get into tight groups when they feel threatened.
They are uncomfortable in small spaces.

Both dunari and humans have eyes that are adapted to their respective home stars. This means
that dunari can not see the short wavelength colors (blue) that humans can (red dwarf stars
produce very little blue light). The dunari refer to such light as ``ultra-green.'' On the other
hand dunari can see a color they call ``rubinum'' that the humans refer to as ``near infrared.''
To dunari eyes human lighting appears yellow-green (no rubinum content) and, because they can't
see the ultra-green light it produces, fairly dark.

On a clear day the sky of Rujar is green\footnote{Earth's sky is blue because small particles
  in the atmosphere scatter the high frequency light more effectively. On Rujar ``high
  frequency light'' is green light.}. However because of Rujar's extensive deserts there is
often a lot of dust in the air causing the sky color to be yellow, or even tan or brown.

Because they do not experience periodic day and night, Rujaran life forms, including the
dunari, do not sleep in the way humans (and most Earth life) does. However, Rujar has a
rotational axis that is inclined to its orbital plan, much as Earth's does, and so there are
seasons on Rujar. Because of the rapid orbital motion these seasons are very short, only
lasting about a week. The dunari are sensitive to this cycle and go into a kind of hibernation
for most of the winter season, lasting about 8 days. It is during this time that their bodies
obtain the rest that humans get during their daily sleep cycle.

Another consequence of this is that the Narlarken moves, depending on latitude, back and forth
during the short Rujaran year. At high latitudes, Rujar does experience periodic day and
night on an annual basis. For example the Rujaran poles spend 18.5 days every year in sunlight
and 18.5 days every year in darkness.

% Physics check: Would a tidally locked planet actually have an inclined rotational axis? It
% seems like the tidal locking might also orient the rotational axis to be perpendicular to the
% orbital plane.

\section{Dunari Time}

Argenia and Forbin use different time and date systems. The information below describes the
Argenian system.

Rujar's orbital period is 37 days, 2 hours, 14 minutes. This is a time the Rujaran people
call a "year." Each year is divided into 120 "hours" and thus each Rujaran hour is equivalent
to 7h25m Earth time. The seasons are: Winter from hour 105 to hour 15, Spring from hour 15 to
hour 45, Summer from hour 45 to hour 75, and Autumn from hour 75 to hour 105. During their
wakeful time the dunari tend to work about two hours and break for one hour. This cycle starts
around hour 20 and goes to hour 100. The dunari do not have a word for the time interval ``day''
or for times related to the day such as ``morning'' or ``afternoon.''

Dunari use ``day'' to refer to the part of their planet that is always in sunlight and ``night''
to refer to the part that is always dark. Thus ``day'' and ``night'' are places, not times.

The dunari divide each hour into 120 ``arnets'' and thus each arnet is equivalent to 3m43s Earth
time. Earth's day is 3.24 Rujaran hours.

Finally the dunari define 120 years as a ``arnox'' (same spelling for both singular and plural
forms). This corresponds to about 12 years, 2 months of Earth time.

The dunari sleep about 1/4 of the year (30 hours Rujaran time or about 9 1/4 Earth days),
generally during the winter.

When reading dunari text a human reader should keep in mind that an ``hour'' is about 8 hours of
Earth time, an ``arnet'' is about four minutes of Earth time, and that about 10 Rujaran years
is a single Earth year. The dunari have a lifespan similar to humans but sometimes the numbers
are a little jarring. For example Chark's young son died of Puget's disease at the age of 70
(only about 7 Earth years). Similarly Tusk's petulant daughter is ``not quite 160,'' meaning
that she is not yet 16 Earth years old.


\section{Dunari Politics}

There are two major nations on Rujar: Argenia and Forbin. They are located on opposite sides
of the planet, each with large population centers in the lush, tropical equatorial regions near
the Narlarken. Argenia is to the east and Forbin is to the west.

Both Argenia and Forbin are very old nations and have dominated dunari politics since ancient
times. There have been, now and then, other smaller nations as well but at the time of the story
they are all historic. Argenia and Forbin, however, are about as different as can be with very
different political, religious, philosophical, and racial backgrounds.

The two countries have a difficult history with many wars and conflicts. As dunari technology
improves the intensity and destructive nature of their wars have also increased. Thus the
countries are in a kind of technological arms race that mirrors the arms race between the USA
and the USSR during the 20th century.

Not long before the action of the story Argenia and Forbin fought the Harkenite War, the first
war that used aerocar technology. The result was destruction and civilian loss of life on an
unprecedented scale. At the time of the story there is an uneasy peace between Argenia and
Forbin and a cold war related to the development of RADAR. The arrival of the humans creates
both tensions and opportunities for the two nations. There are concerns that the humans will
favor one nation over another, but there is also a call for Rujaran solidarity before the
humans.

The story is told from an Argenian point of view and in many respects is more about
Argenian/Forbin relations than it is about Rujaran/Human relations. It is my intention to
specifically not use any Forbin point of view because I want the Forbinites to seem ``strange''
and ``alien'' just as the humans do. Indeed, it is the interplay of these two sets of alien
cultures that gives the story its moral message.

For example the Forbinites are a different race than the Argenians and have visible markings on
their face, trunks, and legs (think: raccoon). This means any Forbinites in Argenia can be
immediately identified as such, creating problems of discrimination, etc.

Argenia is a liberal democracy with a vibrant economy. It is modeled on the United States.
Forbin is a constitutional monarchy with a more precarious economy. It is modeled on the Soviet
Union. Despite Forbin's relatively backward industrial base, it happens that Forbin is well
endowed with natural resources. This tension has been the source for some of the Argenian/Forbin
conflicts of the past (religious differences being another major source of conflict).

Modern dunari society has entered their fossil fuel era. However Rujar has fewer fossil fuel
reserves than Earth did at a similar point in human history. This is because Rujar in general
has far less biomass than Earth. As a result the dunari already realize that sustainability is
going to be a problem for them and this influences their political and social agenda.

In many respects Rujar is ideal for ``alternate'' energy sources. The steady Nermelum and
Nermela winds make wind generation attractive. The continuous sunshine makes solar energy
attractive. However, dunari material science is not yet able to build large windmills that won't
get ripped apart, and dunari scientists have not yet discovered photovoltaics. However there are
some trial solar electric stations under development that work by concentrating sunlight onto a
boiler to create high pressure steam, etc.

Mainstream dunari society hopes that between increasing efficiency and deploying alternate
energy sources they can achieve a sustainable technology. In this respect dunari views on this
matter follow the attitudes prevalent on early 21st century Earth.

Many dunari hope that these energy technologies can be deployed more easily and more effectively
with human help. Indeed, the humans are open to transferring the necessary technology to
Rujar. However the humans have hesitations. Technology can be abused and the dunari have a
long history of bloody war. The humans do not want to be responsible, even indirectly, for
providing weaponizable technology. Thus the official human stance is that they will provide
education and assistance but the process must be gradual so that dunari culture can adapt.

Some dunari are frustrated by the human reticence to help. They see it as patronizing. Some
dunari wonder if the humans are afraid of the dunari becoming dominant if they gain too much
technology; they assume the humans actively want to keep the dunari down. Thus there is some
resentment toward humans in this area.

There are more radical factions of dunari society at play as well. Long before the humans
arrived many dunari concluded that the best path was to return to a simpler, less technically
oriented lifestyle. This attitude is supported by the interpretation of certain religious
writings attributed to Jurita and her Seleck. In mainstream Argenian society a political party,
the Limzar Party, uses this philosophy to guide their platform. The Limzar Party is a minority
party but they are slowly gaining strength and many ordinary dunari are sympathetic to some of
their ``traditional'' positions.

On the fringes of society is the Junar. They are a radical organization dedicated to forcing
Argenia, and ultimately all of Rujar, back to a ``pure'' lifestyle, using violence if
necessary. In effect the Junar is an environmental terrorist organization. They also have
elements of religious fanaticism (the name ``Junar'' literally means ``Of God'' in Old
Argenian).

Argenian authorities have long suspected some sort of relationship between the Limzar Party and
the Junar. Their goals are generally the same and some Limzar leaders have almost endorsed the
Junar on occasion. However no proof of a connection has been found. It turns out there is such a
connection, as we learn; certain well known Limzar supporters are secret Junar leaders. However
the official Limzar Party leadership is not aware of this and does not condone it when they find
out.

The Limzar Party has mixed feelings about the humans. On one hand they applaud the human
reluctance to just hand over technology, but they also object to the human long term plan to do
so eventually. Rachel's presence on Rujar is seen as a step in the wrong direction and some
Limzar spokesmen have said so publicly.

The Junar definitely don't like the humans. They are seen as deeply embedded in ``ungodly''
technology and not to be trusted. The Junar are unimpressed with the human's reluctance to
divulge technology. Instead the Junar assume the humans are just playing the dunari and will
eventually use their technology to overthrow the dunari and extract from it by force whatever
resources are desired.

The Junar are also radical enough to try violence against the humans. Once they become convinced
that humans won't fight (or so they believe) they kidnap Rachel and attempt to execute her on
the theory that the humans will respond by just leaving rather than retaliate. The Junar try to
frame Forbin as well in an effort to incite a war between the two countries. This is intended to
not only to discourage the humans but also to get Argenia and Forbin armed forces to destroy
each other along with as much other technology as possible.

\section{Dunari Religion}

The Argenians following a monotheistic religion based on a Goddess, Jurita, and a group of
seven demi-goddesses called Jurita's Seleck. Together Jurita and her Seleck created the universe
and dictated spiritual law (which, of course, is subject to wide interpretation).

Despite having a religion with entirely female deities, Argenian society is quite male
dominated. Indeed, some of Jurita's Laws seem to dictate this. However, Argenian women
traditionally become part of selecks of their own. Technically these selecks are spiritual
groups with Jurita herself as the spiritual leader of each. In modern Argenia they are mostly
social groups or support groups for the members.

Socially humans don't have anything quite like a seleck. It's intended to be ``more than
friends, less than family.'' Girls typically form selecks with friends or sometimes with others
arranged by their parents, all as part of their coming of age. A woman might be in the same
seleck for life, but more commonly she would be in a number of different ones over the course of
her lifetime.

In Argenia all female names (and only female names) start with `J.' This is a long standing
tradition that honors Jurita.

The Forbinites have a polytheistic religion with a collection of gods and goddesses that engage
in various antics (similarly to the human Greek and Roman gods... personal grudges, affairs,
etc). Women do not figure so prominently in their religion but women are more uniformly
integrated into Forbin society. The Forbinites consider the Argenian religion to be ``bizarre''
and the Argenians consider the Forbinite religion to be ``primitive.''

When Rachel is asked about her religion she is intentionally evasive. As an astro-physicist in
an advanced star-faring civilization her understanding of the structure of the universe is far
deeper and more profound than the Rujaran's. Yet any simple answer that she might give would be
interpreted by the dunari as a denial of Jurita's existence.

However, the Physical Incompleteness Theorem derived in the late 21st century on Earth shows
just the opposite: no logically self-consistent description of reality can explain all true
events. This theorem removes the perceived conflict between religion and science. Unfortunately
the mathematics needed to understand the Physical Incompleteness Theorem is daunting, and Rachel
is unable to explain it quickly and easily to those who ask.
